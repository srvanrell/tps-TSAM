
% This LaTeX was auto-generated from an M-file by MATLAB.
% To make changes, update the M-file and republish this document.

\documentclass{article}
\usepackage{graphicx}
\usepackage{color}

\sloppy
\definecolor{lightgray}{gray}{0.5}
\setlength{\parindent}{0pt}

\begin{document}

    
    
\subsection*{Contents}

\begin{itemize}
\setlength{\itemsep}{-1ex}
   \item Ejercicio 8 - Árboles de decisión
   \item Ejercicio 8.1
   \item crecerArbol.m
   \item impureza.m
   \item Ejercicio 8.2
\end{itemize}


\subsection*{Ejercicio 8 - Árboles de decisión}

\begin{verbatim}
clc; clear all; close all;
\end{verbatim}


\subsection*{Ejercicio 8.1}

\begin{par}
Se tienen datos acerca de la realización o suspensión de partidos de tenis en función del pronóstico del tiempo y los datos del día que se han volcado en el Cuadro 1:
\end{par} \vspace{1em}
\begin{verbatim}
% atributos de cada ejemplo: pronostico, temperatura, humedad, viento
% el ulitmo elemento de cada ejemplo es la clase
datosCuadro1 = {'soleado',  'calor',    'alta',     'no', 'N';
                'soleado',  'calor',    'alta',     'si', 'N';
                'nublado',  'calor',    'alta',     'no', 'P';
                'lluvioso', 'moderado', 'alta',     'no', 'P';
                'lluvioso', 'frio',     'normal',   'no', 'P';
                'lluvioso', 'frio',     'normal',   'si', 'N';
                'nublado',  'frio',     'normal',   'si', 'P';
                'soleado',  'moderado', 'alta',     'no', 'N';
                'soleado',  'frio',     'normal',   'no', 'P';
                'lluvioso', 'moderado', 'normal',   'no', 'P';
                'soleado',  'moderado', 'normal',   'si', 'P';
                'nublado',  'moderado', 'alta',     'si', 'P';
                'nublado',  'calor',    'normal',   'no', 'P';
                'lluvioso', 'moderado', 'alta',     'si', 'N';};
\end{verbatim}
\begin{itemize}
\setlength{\itemsep}{-1ex}
   \item a) Construya "a mano" (haciendo todas las cuentas) y dibuje el árbol binario de decisión que describa los datos sobre juegos de tenis del cuadro. Utilice como impureza la entropía.
\end{itemize}
\begin{par}
El árbol construido a mano se esquematiza en la página siguiente. Los calculos realizados a mano se incluyen a continuación del esquema del árbol.
\end{par} \vspace{1em}
\begin{itemize}
\setlength{\itemsep}{-1ex}
   \item b) Escriba un programa que permita crecer áboles de decisión binarios y utilícelo para generar automáticamente un árbol a partir de los datos anteriores. Contraste con el árbol generado por Ud.
\end{itemize}


\subsection*{crecerArbol.m}

\begin{verbatim}
dbtype crecerArbol.m
\end{verbatim}

        \color{lightgray} \begin{verbatim}
1     function arbol = crecerArbol(datos, ejemplos, listaCaracteristicas,...
2                                  tipoImpureza, debugging)
3     % arbol = crecerArbol(datos,ejemplos,listaCaracteristicas,tipoImpureza,debugging)
4     % construye el arbol a partir de los datos dados
5     %   datos es una celda donde cada ejemplo esta por renglon y por columnas
6     %   las características. La ultima columna corresponde a la clase. Las
7     %   caracteristicas pueden ser cadenas de caracteres o numéricas.
8     %   
9     %   ejemplos es una lista de los ejemplos a utilizar, por defecto se
10    %   utilizan todos si ejemplos es una matriz vacía.
11    %   
12    %   listaCaracterísticas es una celda con los nombres de cada tipo de
13    %   característica
14    %
15    %   tipoImpureza es un string que configura la impureza a utilizar, puede
16    %   ser:
17    %   - 'ent': calcula la impureza por la entropia (por defecto)
18    %   - 'gini': calcula la impureza de Gini
19    %   - 'miss': impureza por la minima probabilidad de errar la clasificación
20    %
21    %   debugging imprime informacion por pantalla:
22    %   - 0: no imprime nada
23    %   - 1: imprime la información de cada nodo del arbol generado
24    %   - 2: igual que 1 más información intermedia utilizada en el crecimiento
25    %        del árbol
26    %
27    %   arbol es una celda donde cada elemento es un nodo. Cada nodo tiene
28    %   distinto tipo de información de acuerdo a su tipo (por ejemplo, id,
29    %   pregunta, hijos, clase)
30    %   clase 
31    if (nargin < 2) || isempty(ejemplos)
32        n1.ejemplos = 1:size(datos,1); % inicializo con todos los ejemplos
33    else
34        n1.ejemplos = ejemplos; % inicializo con los ejemplos dados
35    end
36    if (nargin < 3) || isempty(listaCaracteristicas)
37        auxc = 1:(size(datos,2)-1); % todas las caracteristicas
38        listaCaracteristicas = sprintf('Atributo%i ',auxc);
39        listaCaracteristicas = strsplit(listaCaracteristicas(1:end-1));
40    end
41    if (nargin < 4) || isempty(tipoImpureza)
42        tipoImpureza = 'ent'; % calcula impureza por entropia
43    end
44    if nargin < 5
45        debugging = false; % Por defecto no imprime por pantalla
46    end
47    
48    
49    n1.id = 1;                     % id del nodo raiz
50    nodosARevisar = n1.id;         % el primer nodo a revisar es el nodo raiz
51    n1.tipo = 'aRevisar';          % tipo provisorio
52    arbol{1} = n1;                 % arbol inicial con nodo raiz
53    
54    % mientras queden nodos por revisar
55    while ~isempty(nodosARevisar)
56        % Reviso el siguiente nodo de la lista
57        nodo.id = nodosARevisar(1);
58        % enumero los ejemplos que le corresponden 
59        ejemplos = arbol{nodosARevisar(1)}.ejemplos;
60        
61        % Genera una lista con las clases encontradas en los ejemplos
62        clasesPorEjemplo = datos(ejemplos,end); %lista la clase de cada ejemplo
63        clases = unique(clasesPorEjemplo);      %lista con las posibles clases
64    
65        %% Si todos los ejemplos pertenecen a la misma clase creo un nodo hoja
66        if length(clases) == 1
67            nodo.tipo = 'hoja';             % tipo de nodo
68            nodo.clase = clases{1};         % clase del nodo hoja
69    
70        %% Si no son todos los ejemplos de la misma clase se debe elegir que 
71        %  caracteristica utilizar
72        else
73            % Cantidad de ejemplos e impureza del nodo padre
74            nPadre = length(ejemplos);                 
75            iPadre = impureza(clasesPorEjemplo, tipoImpureza); 
76            deltaimax = 0; %variable auxiliar para buscar la mejor ramificaci�n
77            nodo.tipo = 'sinAsignar'; % etiqueta de control interno
78            
79            % Para cada caracteristica candidata
80            for lc = 1:length(listaCaracteristicas)
81                % si las caracteristicas son num�ricas
82                auxcarac = horzcat(datos{ejemplos,lc});
83                esNumerica = isnumeric(auxcarac);
84                
85                % exploro las alternativas de cada caracteristica
86                if esNumerica
87                    % si es num�rica genero los candidatos ordenando por valor
88                    % y tomando el valor medio en cada salto de clase
89                    [valor, orden] = sort(horzcat(datos{ejemplos,lc}));
90                    clasesOrdenadas = clasesPorEjemplo(orden);
91                    decisionesCandidatas = [];
92                    for co = 2:length(clasesOrdenadas)
93                        if ~strcmp(clasesOrdenadas(co-1),clasesOrdenadas(co))
94                            vmedio = 0.5 * (valor(co-1) + valor(co));
95                            decisionesCandidatas=[decisionesCandidatas vmedio];
96                        end
97                    end
98                else
99                    % si es categ�rica cada valor se vuelve candidato
100                   decisionesCandidatas = unique(datos(ejemplos,lc));
101               end
102               iterFinal = length(decisionesCandidatas); % itero sobre todas 
103                                                         % las opciones
104               
105               % si hay dos valores posibles no necesita iterar m�s de una vez
106               if (iterFinal == 2) && esNumerica
107                   iterFinal = 1;
108               end
109               
110               % Para cada valor posible de la caracteristica candidata se calcula
111               % el deltai, a la vez que se compara con el maximo almacenado.
112               for dc = 1:iterFinal
113                   if esNumerica
114                       auxHijoSi = (decisionesCandidatas(dc) > auxcarac);
115                   else
116                       auxHijoSi = strcmp(decisionesCandidatas(dc), datos(ejemplos,lc));
117                   end
118                   auxHijoNo = ~auxHijoSi;
119                   nHijoSi = sum(auxHijoSi);
120                   nHijoNo = sum(auxHijoNo);
121                   iHijoSi = impureza(clasesPorEjemplo(auxHijoSi), tipoImpureza);
122                   iHijoNo = impureza(clasesPorEjemplo(auxHijoNo), tipoImpureza);
123                   
124                   deltai = iPadre - (nHijoSi/nPadre) * iHijoSi - ...
125                       (nHijoNo/nPadre) * iHijoNo;
126                      
127                   % Si deltai es mayor al maximo pasa a ser la decision elegida
128                   if deltai > deltaimax
129                       deltaimax = deltai;
130                       
131                       % El nodo actual se divide en dos.
132                       nodo.tipo = 'divisor';
133                       % Se guarda la pregunta utilizada para dividir
134                       if esNumerica
135                           nodo.pregunta = ['¿' listaCaracteristicas{lc} ...
136                                            sprintf(' < %0.1f?', decisionesCandidatas(dc))];
137                       else
138                           nodo.pregunta = ['¿' listaCaracteristicas{lc} ...
139                                            '=' decisionesCandidatas{dc} '?'];
140                       end
141                       % Se asignan los id de los nodos que responden si y no
142                       nodo.HijoSi = length(arbol) + 1;
143                       nodo.HijoNo = length(arbol) + 2;
144                       % Tipo provisorio
145                       nodoHijoSi.tipo = 'aRevisar';
146                       nodoHijoNo.tipo = 'aRevisar';
147                       % Los ejemplos del padre se dividen entre los nodos hijos
148                       nodoHijoSi.ejemplos = ejemplos(auxHijoSi);
149                       nodoHijoNo.ejemplos = ejemplos(auxHijoNo);
150                   end
151                   
152                   if debugging > 1
153                       if esNumerica
154                           preg = ['¿' listaCaracteristicas{lc} ...
155                                            sprintf(' < %0.1f?', decisionesCandidatas(dc))];
156                       else
157                           preg = ['¿' listaCaracteristicas{lc} ...
158                                            '=' decisionesCandidatas{dc} '?'];
159                       end
160                       fprintf('pregunta: %s', preg)
161                       fprintf('deltai: %0.5f\n', deltai)
162                   end
163               end
164           end
165       end
166       
167       % Si no se pudo ramificar y el nodo contiene m�s de una clase
168       if strcmp(nodo.tipo,'sinAsignar')
169           cuentaXclase = histClases(clasesPorEjemplo); % simil histograma
170           [~, imax] = max(cuentaXclase);  % indice de la clase mayoritaria
171           nodo.clase = clases{imax};      % clase mayoritaria
172           nodo.tipo = 'hoja';             % tipo de nodo
173       end
174       
175       % Si est� activo el debugging imprimo en pantalla
176       if debugging > 1
177           fprintf('\nEn nodo %i reviso los ejemplos:\n',nodo.id);
178           fprintf('%i, ',ejemplos);
179           
180           if strcmp(nodo.tipo,'hoja')
181               fprintf('\nNodo hoja, clase: %s\n', nodo.clase);
182               if length(clases) > 1
183                   fprintf('------¡Impuro!----------\n')
184                   fprintf('%s\t', clases{:});
185                   fprintf('\n');
186                   fprintf('%i\t', cuentaXclase);
187                   fprintf('\n');
188               end
189           elseif strcmp(nodo.tipo,'divisor')
190               fprintf('\nNodo divisor: %s\n', nodo.pregunta);
191           else
192               disp('---- ¡Ups! el árbol tiene una rama seca ----');
193           end
194       end
195       
196       % Definido el nodo lo agrego al arbol
197       arbol{nodosARevisar(1)} = nodo;
198       
199       % Si es un nodo divisor agrego los hijos a la lista para revisar
200       if strcmp(nodo.tipo, 'divisor')
201           arbol{nodo.HijoSi} = nodoHijoSi;
202           arbol{nodo.HijoNo} = nodoHijoNo;
203           nodosARevisar = [nodosARevisar nodo.HijoSi nodo.HijoNo];
204       end
205       nodosARevisar = setdiff(nodosARevisar, nodosARevisar(1));
206       
207       % Limpieza antes del siguiente ciclo
208       clear nodo nodoHijoSi nodoHijoNo; 
209   end
210   
211   % Si está activo el debugging imprime el árbol creado
212   if debugging
213       disp('-------------------------')
214       disp('Detalles del árbol creado')
215       disp('-------------------------')
216       for aa = 1:length(arbol)
217           disp(arbol{aa})
218       end
219   end
220   
221   end
\end{verbatim} \color{black}
    

\subsection*{impureza.m}

\begin{verbatim}
dbtype impureza.m
\end{verbatim}

        \color{lightgray} \begin{verbatim}
1     function imp = impureza( clasesEnEjemplos, metodo)
2     % Calcula la impureza del nodo dada la lista de clases por ejemplos. 
3     %   imp = impureza( clasesEnEjemplos, metodo)
4     %   clasesEnEjemplos: lista con la clase de cada ejemplo
5     %   imp: impureza calculada
6     %   metodo: utilizado para calcular la impureza, definido por un string:
7     %   - 'ent': calcula la impureza por la entropia.
8     %   - 'gini': calcula la impureza de Gini
9     %   - 'miss': impureza por la minima probabilidad de errar la clasificación
10    
11    cuentas = histClases(clasesEnEjemplos); % cantidad de ejemplos por clase
12    total = sum(cuentas);                   % cantidad de ejemplos en el nodo
13    imp = 0;                                % inicializo la impureza
14    
15    % Si no son todos de la misma clase
16    if all(cuentas ~= total)
17        if strcmp(metodo, 'ent')
18            imp = - sum( (cuentas./total) .* log(cuentas./total) );
19        elseif strcmp(metodo, 'gini')
20            imp = 1 - sum((cuentas./total).^2);
21        elseif strcmp(metodo, 'miss')
22            imp = 1 - max(cuentas./total);
23        else
24            disp('------ Perdón, no sé calcular esa impureza :( ------');
25        end
26    end
27    
28    end
29    

\end{verbatim} \color{black}
    \begin{verbatim}
% listaEjemplos = 1:14; uso todos, no hace falta
nombresCaract = {'pronostico', 'temperatura', 'humedad', 'viento'};
debug = 1; % solo publica el arbol construido
tipoImpureza = 'ent';
arbol1b = crecerArbol(datosCuadro1, [],nombresCaract,tipoImpureza,debug);
\end{verbatim}

        \color{lightgray} \begin{verbatim}-------------------------
Detalles del árbol creado
-------------------------
          id: 1
        tipo: 'divisor'
    pregunta: '¿pronostico=nublado?'
      HijoSi: 2
      HijoNo: 3

       id: 2
     tipo: 'hoja'
    clase: 'P'

          id: 3
        tipo: 'divisor'
    pregunta: '¿humedad=alta?'
      HijoSi: 4
      HijoNo: 5

          id: 4
        tipo: 'divisor'
    pregunta: '¿pronostico=lluvioso?'
      HijoSi: 6
      HijoNo: 7

          id: 5
        tipo: 'divisor'
    pregunta: '¿viento=no?'
      HijoSi: 8
      HijoNo: 9

          id: 6
        tipo: 'divisor'
    pregunta: '¿viento=no?'
      HijoSi: 10
      HijoNo: 11

       id: 7
     tipo: 'hoja'
    clase: 'N'

       id: 8
     tipo: 'hoja'
    clase: 'P'

          id: 9
        tipo: 'divisor'
    pregunta: '¿pronostico=lluvioso?'
      HijoSi: 12
      HijoNo: 13

       id: 10
     tipo: 'hoja'
    clase: 'P'

       id: 11
     tipo: 'hoja'
    clase: 'N'

       id: 12
     tipo: 'hoja'
    clase: 'N'

       id: 13
     tipo: 'hoja'
    clase: 'P'

\end{verbatim} \color{black}
    \begin{par}
Este arbol da igual que el que hice a mano, sólo cambia la forma de expresar las preguntas. Por ejemplo, en nodo 4 (D) en vez de preguntar por ¿soleado? pregunta por ¿lluvioso? que es equivalente porque hay dos posibilidades en ese momento (ya se había preguntado por nublado). En nodo 9 (I) donde podría haber elegido temperatura también elige pronóstico
\end{par} \vspace{1em}
\begin{par}
En la página siguiente esquematizo el árbol encontrado
\end{par} \vspace{1em}
\begin{itemize}
\setlength{\itemsep}{-1ex}
   \item c) Modifique el tipo de impureza empleada y comente las diferencias con el árbol generado en el punto anterior.
\end{itemize}
\begin{verbatim}
tipoImpureza = 'gini';
arbol1cgini = crecerArbol(datosCuadro1, [],nombresCaract,tipoImpureza,debug);
\end{verbatim}

        \color{lightgray} \begin{verbatim}-------------------------
Detalles del árbol creado
-------------------------
          id: 1
        tipo: 'divisor'
    pregunta: '¿pronostico=nublado?'
      HijoSi: 2
      HijoNo: 3

       id: 2
     tipo: 'hoja'
    clase: 'P'

          id: 3
        tipo: 'divisor'
    pregunta: '¿humedad=alta?'
      HijoSi: 4
      HijoNo: 5

          id: 4
        tipo: 'divisor'
    pregunta: '¿pronostico=lluvioso?'
      HijoSi: 6
      HijoNo: 7

          id: 5
        tipo: 'divisor'
    pregunta: '¿viento=no?'
      HijoSi: 8
      HijoNo: 9

          id: 6
        tipo: 'divisor'
    pregunta: '¿viento=no?'
      HijoSi: 10
      HijoNo: 11

       id: 7
     tipo: 'hoja'
    clase: 'N'

       id: 8
     tipo: 'hoja'
    clase: 'P'

          id: 9
        tipo: 'divisor'
    pregunta: '¿pronostico=lluvioso?'
      HijoSi: 12
      HijoNo: 13

       id: 10
     tipo: 'hoja'
    clase: 'P'

       id: 11
     tipo: 'hoja'
    clase: 'N'

       id: 12
     tipo: 'hoja'
    clase: 'N'

       id: 13
     tipo: 'hoja'
    clase: 'P'

\end{verbatim} \color{black}
    \begin{par}
El árbol calculado con la impureza de gini es igual que el hecho a mano y que el del punto 1.b
\end{par} \vspace{1em}
\begin{verbatim}
tipoImpureza = 'miss';
arbol1cmiss = crecerArbol(datosCuadro1, [],nombresCaract,tipoImpureza,debug);
\end{verbatim}

        \color{lightgray} \begin{verbatim}-------------------------
Detalles del árbol creado
-------------------------
          id: 1
        tipo: 'divisor'
    pregunta: '¿pronostico=soleado?'
      HijoSi: 2
      HijoNo: 3

          id: 2
        tipo: 'divisor'
    pregunta: '¿humedad=alta?'
      HijoSi: 4
      HijoNo: 5

          id: 3
        tipo: 'divisor'
    pregunta: '¿temperatura=frio?'
      HijoSi: 6
      HijoNo: 7

       id: 4
     tipo: 'hoja'
    clase: 'N'

       id: 5
     tipo: 'hoja'
    clase: 'P'

          id: 6
        tipo: 'divisor'
    pregunta: '¿pronostico=lluvioso?'
      HijoSi: 8
      HijoNo: 9

       id: 7
     tipo: 'hoja'
    clase: 'P'

          id: 8
        tipo: 'divisor'
    pregunta: '¿viento=no?'
      HijoSi: 10
      HijoNo: 11

       id: 9
     tipo: 'hoja'
    clase: 'P'

       id: 10
     tipo: 'hoja'
    clase: 'P'

       id: 11
     tipo: 'hoja'
    clase: 'N'

\end{verbatim} \color{black}
    \begin{par}
El árbol calculado con la impureza de clasificar mal difiere de los anteriores desde la primera pregunta. Tiene dos nodos menos y tiene una hoja impura (nodo 7), al contrario de los anteriores los grupos que se formaron al clasificar los ejemplos son distintos en su mayoria
\end{par} \vspace{1em}
\begin{par}
En la página siguiente esquematizo el árbol encontrado
\end{par} \vspace{1em}


\subsection*{Ejercicio 8.2}

\begin{par}
Se han modificado los datos de juegos de tenis anteriores para incluir valores numéricos de temperatura y humedad de acuerdo con el Cuadro 2: Modifique el programa desarrollado para tratar con atributos numéricos y genere un nuevo árbol con los datos del cuadro.
\end{par} \vspace{1em}
\begin{verbatim}
datosCuadro2 = {'soleado',  85,    85,   'no', 'N';
                'soleado',  80,    90,   'si', 'N';
                'nublado',  83,    78,   'no', 'P';
                'lluvioso', 70,    96,   'no', 'P';
                'lluvioso', 68,    80,   'no', 'P';
                'lluvioso', 65,    70,   'si', 'N';
                'nublado',  64,    65,   'si', 'P';
                'soleado',  72,    95,   'no', 'N';
                'soleado',  69,    70,   'no', 'P';
                'lluvioso', 75,    80,   'no', 'P';
                'soleado',  75,    70,   'si', 'P';
                'nublado',  72,    90,   'si', 'P';
                'nublado',  81,    75,   'no', 'P';
                'lluvioso', 71,    80,   'si', 'N';};


tipoImpureza = 'ent';
arbol2 = crecerArbol(datosCuadro2, [],nombresCaract,tipoImpureza,debug);
\end{verbatim}

        \color{lightgray} \begin{verbatim}-------------------------
Detalles del árbol creado
-------------------------
          id: 1
        tipo: 'divisor'
    pregunta: '¿pronostico=nublado?'
      HijoSi: 2
      HijoNo: 3

       id: 2
     tipo: 'hoja'
    clase: 'P'

          id: 3
        tipo: 'divisor'
    pregunta: '¿temperatura < 77.5?'
      HijoSi: 4
      HijoNo: 5

          id: 4
        tipo: 'divisor'
    pregunta: '¿temperatura < 73.5?'
      HijoSi: 6
      HijoNo: 7

       id: 5
     tipo: 'hoja'
    clase: 'N'

          id: 6
        tipo: 'divisor'
    pregunta: '¿viento=no?'
      HijoSi: 8
      HijoNo: 9

       id: 7
     tipo: 'hoja'
    clase: 'P'

          id: 8
        tipo: 'divisor'
    pregunta: '¿temperatura < 71.0?'
      HijoSi: 10
      HijoNo: 11

       id: 9
     tipo: 'hoja'
    clase: 'N'

       id: 10
     tipo: 'hoja'
    clase: 'P'

       id: 11
     tipo: 'hoja'
    clase: 'N'

\end{verbatim} \color{black}
    \begin{par}
La función presentada anteriormente ya contiene los cambios realizados para tratar con atributos numéricos. No repito aquí el código por esa causa.
\end{par} \vspace{1em}
\begin{par}
En este caso no usa el pronóstico más allá del nodo raíz. La humedad no la usa nunca. La temperatura la usa 3 veces, aprovechando que tiene más cortes candidatos. El viento una sola vez. El árbol es más profundo en este caso puede ser que se esmere por ir separanado en hojas y termine utilizando más decisiones de las necesarias, típico de busqueda en profundidad. Es más desbalanceado que en los casos anteriores, quizá por tener más flexibilidad de elección prefiere ir dejando nodos hojas puros en el camino.
\end{par} \vspace{1em}



\end{document}
    
