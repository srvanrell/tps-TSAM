\documentclass[11pt,a4paper,final]{article}

\usepackage[utf8]{inputenc}     % Codificación del archivo fuente en UTF8
\usepackage[T1]{fontenc}        % Incorpora fuente con acentos
\usepackage{amsmath}            % Amplía opciones para las ecuaciones
%\usepackage[margin=2cm]{geometry}

% -- Bibliografia --
%\usepackage[style=ieee,
%            backend=bibtex8,
%            maxcitenames=2,
%            mincitenames=1]
%            {biblatex}          % Control sobre las citas y referencias 
%  \addbibresource{Referencias.bib} % Archivo con las referencias
%\usepackage{url}                % Para incluir url clickeables

% -- Tuneando los captions --
\usepackage[margin=10pt,
            font=small,
            labelfont=bf,
            labelsep=endash]
            {caption}

% -- Tablas --
%\usepackage{booktabs}       % Decorado de tablas
%  \heavyrulewidth=1pt       % Lineas gruesas
%\usepackage{tabu}           % Permite control del ancho relativo entre columnas

% -- Figuras --
\usepackage{graphicx}		    % Permite incluir imágenes
  \graphicspath{{Figuras/}}	    % Ruta relativa donde buscar las imágenes
\usepackage[font=small,
            labelfont=bf, 
            subrefformat=parens]
            {subcaption}
            
% -- Incorporar imágenes de Matlab --
\usepackage{pgfplots}
\pgfplotsset{compat=newest} 
\pgfplotsset{plot coordinates/math parser=false}
\usepackage{tikz}
\usetikzlibrary{plotmarks} % para plotear las graficas que vienen de Matlab
            
%\usepackage{color}
%
%\sloppy
\definecolor{lightgray}{gray}{0.8} % para la grafica 2.3

%---------------------------------------------------------------------------------
% DEPRECATED COMMANDS
%---------------------------------------------------------------------------------  
\usepackage[spanish]{babel}     % Configura en modo español a muchas cosas
\usepackage{amsfonts}
\usepackage{amssymb}
\usepackage[babel,
            spanish=spanish]
            {csquotes}          % Comillas francesas en la bibliografia
%---------------------------------------------------------------------------------



\author{Sebastián R. Vanrell\\[3em]}
\title{{\large Curso:}\\\medskip
       {\Large Tópicos Selectos en Aprendizaje Maquinal}\\[3em]
       \textsf{Guía de Trabajos Prácticos Nº1}\\
       \textsf{\large (Ejercicio 9: SVM)}\\\bigskip
       \textsf{Algoritmos para Reconocimiento de Patrones}\\[3em]}
\date{Doctorado en Ingeniería\\\bigskip
      Facultad de Ingeniería y Ciencias Hídricas\\\bigskip
      Universidad Nacional del Litoral \\[5em]
      \today}

\usepackage{color}
\definecolor{lightgray}{gray}{0.5}
\setlength{\parindent}{0pt}

\begin{document}
\renewcommand{\tablename}{Tabla}

\maketitle
\newpage


\section*{Scripts de Matlab del ejercicio 9}

\subsection*{9.1 cálculo de los alfas del problema linealmente separable}

\begin{small}
\begin{verbatim}
% Se define la funcion objetivo
H = [-1  0 -1;
      0  0  0;
     -1  0 -1];
H = (-1) .* H;      % (-1) para pasar de maximizacion a minimizacion

f = (-1).* [1 1 1]; % (-1) para pasar de maximizacion a minimizacion

% Se definen las restricciones
A = (-1) .* eye(3);
b = [0; 0; 0];
Aeq = [1 -1 -1];
beq = 0;

% Se obtienen los alfas
alfa91 = quadprog(H,f,A,b,Aeq,beq)

alfa91 =

    2.0000
    2.0000
         0

% x = quadprog(H,f,A,b,Aeq,beq) resuelve el siguiente problema:
%
%  min 1/2*x'*H*x + f'*x
%
%  sujeto a:   A*x <= b
%            Aeq*x = beq
\end{verbatim}
\end{small}
    
\newpage

\subsection*{9.2 Cálculo de los alfas del problema no linealmente separable}

\begin{small}
\begin{verbatim}
% Se define la funcion objetivo
H = [-4  1  0;
      1 -1  1;
      0  1 -4];
H = (-1) .* H;      % (-1) para pasar de maximizacion a minimizacion

f = (-1).* [1 1 1]; % (-1) para pasar de maximizacion a minimizacion

% Se definen las restricciones
C = 1000000;
A = [(-1).*eye(3); eye(3)];
b = [0; 0; 0; C ; C; C];
Aeq = [1 -1 1];
beq = 0;

% Se obtienen los alfas
alfa92 = quadprog(H,f,A,b,Aeq,beq)
\end{verbatim}
\end{small}

\begin{small}
\begin{verbatim}
alfa92 =

    1.0000
    2.0000
    1.0000

% x = quadprog(H,f,A,b,Aeq,beq) resuelve el siguiente problema:
%
%  min 1/2*x'*H*x + f'*x
%
%  sujeto a:   A*x <= b
%            Aeq*x = beq
\end{verbatim}
\end{small}

\end{document}
